
\documentclass[conference]{IEEEtran}

\usepackage[english]{babel}
\usepackage[utf8x]{inputenc}
\usepackage{amsmath}
\usepackage{graphicx}
\usepackage[colorinlistoftodos]{todonotes}
\usepackage{hyperref}
\usepackage{import}
\usepackage{cite}
\usepackage{lscape}
\usepackage{multirow}
%\usepackage{dblfloatfix}    % To enable figures at the bottom of page



%\bibliographystyle{ieeetr}
%\providecommand{\abs}[1]{\lvert#1\rvert}


\title{Parashat: Bo\\ \textit{Ve}}

% for over three affiliations, or if they all won't fit within the width
% of the page, use this alternative format:
% 
\author{\IEEEauthorblockN{Kunst, James Jules Waldemar}
\IEEEauthorblockA{Bet-Melej Haba}}
%\IEEEauthorblockA{waldemark89@hotmail.com}}


\begin{document}
\maketitle


\section{Introduction}
\begin{itemize}
\item
Shemot 10:1–13:28
\item 
Jeremias 46:13-28
\end{itemize}

Resumen
\begin{itemize}
\item 
Plaga de las langostas
\item
Plaga de la oscuridad
\item
Plaga de la muerte de los primogentos
\item
Ofrenda de Pesaj e instrucciones para Pesaj
\item
Salida de egipto
\item
Haftara \\
Profecia de Nabuconodosor. Palabras de esperanza para Yakov, es castigado pero no destruido.
\end{itemize}


\section{Preguntas}

\begin{enumerate}
\item ¿Que diferencias hay entre las 7 primeras plagas y las 3 ultimas?
\end{enumerate}



\section{Desarrollo}
\begin{enumerate}
 \item Las primetas 7 plagas D's le demuestra a Paro que es el verdadero D's sobre todo los fisico. A partir de la plaga numero 8, Paro ya reconocio que el D's de los Israelitas es el unico y verdadero D's. Entonces si se niega a obedecer conociendo la verdad, esta pecando a plena  conciencia, esta desobedeciendo, no pecando por pagano. 
 
Durante las primeras 7 plagas D's le dice a Mashe que va a envalontar a Paro, es decir, le va a dar coraje. Le va a inflar el orgullo.
Pero en las ultimas 3, dice que va a endurecer su corazon, lo va a hacer terco. 
¿Pero como hace esto D's? . ¿Acaso manipula sobrenaturalmente las emociones de Paro?

No, sino que lo hace de una forma muy simple, tocando su Ego. ¿ Como hace esto?, atravez de las palabras de Moshe. 


\textit{Shemot 10:3 ---- }

Moshe le habla de una forma muy fuerte, una forma como nunca antes le habia hablado y sumado a eso, en Shemot 10:8
 
 
 
\textit{Shemot 10:8 ---- }


Moshe toma el control politico de Egipto, es decir los sirvientes de Paro se ponen del lado de Moshe. 

Todo esto toca el ego de Paro, lo estan considerando alguien humano, Moshe comienza a ganarse el corazon de sus subditos.

Fijense en el texto, Paro intenta ganar un poco de control ante lo que le dice Moshe ante sus servidores.

Todo esto toca el ego de Paro y es lo que lo pone duro interiormente.


Y esta es una de las enseñanzas mas complejas de las se puede aprender en cuanto al relato de la Pascua. 


El problema con Paro. 

\end{enumerate}




\section{Conclusion}







\bibliographystyle{IEEEtran}
\bibliography{reference}





\end{document}
