
\documentclass[conference]{IEEEtran}

\usepackage[english]{babel}
\usepackage[utf8x]{inputenc}
\usepackage{amsmath}
\usepackage{graphicx}
\usepackage[colorinlistoftodos]{todonotes}
\usepackage{hyperref}
\usepackage{import}
\usepackage{cite}
\usepackage{lscape}
\usepackage{multirow}
%\usepackage{dblfloatfix}    % To enable figures at the bottom of page



%\bibliographystyle{ieeetr}
%\providecommand{\abs}[1]{\lvert#1\rvert}


\title{Parashat: Haazinu \\ \textit{Escuchen}}

% for over three affiliations, or if they all won't fit within the width
% of the page, use this alternative format:
% 
\author{\IEEEauthorblockN{Kunst, James Jules Waldemar}
\IEEEauthorblockA{Bet-Melej Haba}}
%\IEEEauthorblockA{waldemark89@hotmail.com}}


\begin{document}
\maketitle


\section{Introduction}



\section{Preguntas}

\begin{enumerate}
\item

Las ultimas 3 parasha de la Torah que son Vaieji, Haazinu y V-zot HaBeraja  , son consideradas como el cierre de la Torah. La conclusion  de la Torah. ¿Por que? ¿Que es una conclusion, y que nivel de importancia tiene?.¿ Porque la conclusion pareceria hablar de unicamente de Moshe, cuando todos sabemos que si bien el es importante, mas importante es el pueblo.?
\item
¿Porque no sabemos donde esta enterrado Moshe?. ¿Porque la Torah lo quiso asi?
\item
Otra pregunta es acerca de la ultima parasha que es la que sigue, que se llama " V-zot HaBeraja" que significa y "esta es la bendicion". Esta parasha pasa casi siempre desapercibida. ¿Por que?.  ¿Como es posible que siendo la ultima parasha, sea  la  menos estudiada?
\end{enumerate}



\section{Desarrollo}
\begin{enumerate}
\item
Conclusión es muy importante, es el cierre del libro, ya sea malo o bueno, es la ultima enseñanza.
¿Como termina la Torah? ¿Seria su ultima enseñanza?
Si leemos la ultima oracion :
\begin{itemize}
\item Dev 34.10-12 :  Moshe es el sujeto de la ultima oracion 
\end{itemize}
La ultima sentencia se enfoca en Moshe. 
La supuesta conclusion de la Torah, habla de la tristeza de Moshe al no poder entrar a la tierra. 
Habla de la muerte de Moshe. Y de como el Pueblo queda desamparado. 


Y nos deja una pregunta muy problemática. ¿Como es que la Torah habla de Moshe como el mas grande profeta, sin embargo no puede entrar a la tierra?.
Se nombra la promesa que se relata durante los 5 libros después de cientos de años.... y al final 
le que no cruzara. 
Como lo exalta tanto a Moshe y luego no lo deja entrar.
Es como hacerle desar un dulce a un niño. es muy caotico que la Torah termine de esta manera. y es muy difcil de entender  porque?.¿ Y es acaso esa la ultima enseñanza?
¿Donde esta la conclusión inspiradora de la Torah después de terminar asi?
Para responder todo esto, nos hacemos una nueva pregunta. 
\begin{itemize}
\item
¿Que tal si la conclusion esta antes y no al final ?
Al final de Parasha Haazinu...
Luego del canto en Dev 32:46:47 --- inspirador..."Es vida"
y luego se va... y comienza la ultima Parasha.
\end{itemize}
Eso si que parece una conclusion valida.
Y ademos como la torah siempre tiene algo oculto para decirnos. hay un detalle en el versiculo 
 Dev 32:48.
%Dev 34:4 
%Muerte de MOshe y no se sabe donde esta.... y luego Josue recibe el espiritu.
%entonces si tenemos un mensje inspirador. pero luego se habla de moshe otra vez....
%Acaso no esta de mas nombrar a Moshe otra vez? .. porque insiste en MOshe..
%Epilogo... donde esta el epilogo en la TORA? 
%los ultimos 3 capitulos se tratan de la muerte de moshe..
%debemos ver de cerca la antes de estos capitulos. ----- Frase " En ese mismo dia"..-- porque aparece esta palabra?




\subsection{Etapas}
\begin{itemize}
\item Si analizamos un poco mas el texto podemos notar algo extraño. Existe una frase en Dev 32:48 que dice asi.
"...en ese mismo dia..." Esta oracion es tan rara que solo se usa 3 veces mas en la Torah. 
\begin{itemize}
\item Bereshit 7:13 "...En este mismo día entraron Noé, y Sem, Cam y Jafet hijos de Noé, la mujer de Noé, y las tres mujeres de sus hijos, con él en el arca;"  --- El arca se cierra y comienza una nueva etapa.
\item Bereshit 17:26 "...En el mismo día fué circuncidado Abraham é Ismael su hijo." --- D's acnuncia a Abraham que va a tener un hijo, se circuncida y entra en el pacto de D's. Comienza una nueva etapa.
\item Shemot 12:41 "...	Y sucedió que al cabo de los cuatrocientos treinta años, en aquel mismo día, todos los ejércitos del Señor salieron de la tierra de Egipto." --- El pueblo de Israel sale de Egipto. Comienza una nueva etapa.
\end{itemize}
\end{itemize}

Es decir, esta combinación de palabras, siempre se usaron en la Torah para marcar el inicio de una nueva etapa. 
Y justamente eso es lo que se esta marcando en esta Parasha. La muerte  de Moshe, es triste si, pero marca una nueva etapa. 
Una etapa donde el pueblo va a poner en practica todo lo que aprendió en el desierto. Y a entender el concepto de que la Torah no esta en el Cielo, y no hace falta un profeta del nivel de Moshe para acceder a ella, sino que esta en ellos.
Y fijense la importancia de cada uno de los  acontecimientos, son puntos de quiebre para el mundo, donde se finaliza una etapa y se comienza una nueva.
\begin{itemize}
\item
El etapa del mundo pecador de Noe.
\item
El etapa de la torre de Babel.
\item
El etapa de Egipto.
\item
La etapa del desierto.
\end{itemize}

El mundo no es el mismo luego de que se dice esta frase.
Lo que sucede en esta parasha es tan importante como esos acontecimientos. 
La Torah usa esa frase para marcar la importancia de este evento. 
Haazinu marca una nueva etapa. La cual paradojicamente marca algo triste como lo es la muerte de Moshe como algo lindo , que es el cumplimiento de la promesa en la entrega de la tierra prometida.

Que tan importante fue la perdida de Moshe.?
Gemara -- cuenta que se perdieron 3000 leyes, se olvidaron . pero fueron recuperadas atraves del analisis de Kaleb.
Es decir ahora son ellos los que las guardan  y tienen miedo de perderlas.. entraron en panico.


La muerte de Moshe fue anticipada por el mismo:

\begin{itemize}
\item
Dev 30:12:13   "...no esta en el cielo... "
\end{itemize}

cuando les dice que la torah no esta lejos de ellos ni al otro lado del rio , justo como sucede ahora. Moshe muere al otro lado del rio. Y ellos se quedan a cargo de guardar y preservar los mandamientos.

Si bien Moshe fue muy importante para el pueblo de Israel. Por la debilidad del pueblo ,
Moshe tambien interferia en la relacion del Pueblo con D's. Ellos muchas veces se refugiaban en el y se quejaban con el..
como si fuera un semi D's. Como si el poder fuera de el. Es por eso que en necesario que Moshe tambien parta.

\item ¿Y que pasa con la ubicación del cuerpo de Moshe? Justamente por la importancia que tuvo como guia del pueblo, Israel fácilmente podría internar usar su tumba como un santuario y querer tener a Moshe como intermedario, como un santo. 
Imagínense que en 2 Reyes 18:4 se lee como el Pueblo todavia idolatraba a la serpiente que Moshe levanto en el desierto. Imagínense que hubiera sucedido si hubieran sabido donde estaba enterrado Moshe.

\item 
Por ultimo tenemos que responder sobre la ultima de las ultimas parasha. " V-zot HaBeraja" que significa y "esta es la bendición". Esta parasha pasa desapercibida por varias cosas, una de ellas es que es que su redaccion es diferente, osea no sigue la misma dinamica que las demas parashot y otra que por tradición, se lee junto con Bereshit. Y al ser Bereshit un libro tan lleno de significado y simbologia, esta parasha queda en el  fondo de todo y no recibe la atención, que se merece. 
Y esta parasha es tan inmportate porque es en la cual Moshe muere y 

\end{enumerate}

\section{Conclusion}
Al igual que sucedio con Noe, Abraham, el pueblo de Israel al salir de Egipto , la muerte de Moshe y el ingreso a la Tierra prometida. Justamente Yom Kipur marca el comienzo de una nueva etapa. Un comienzo donde nosotros estamos presentes. del cual podemos ser parte y posee la misma importancia. Un nuevo comienzo.






\bibliographystyle{IEEEtran}
\bibliography{reference}





\end{document}
