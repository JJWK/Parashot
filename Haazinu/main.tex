
\documentclass[conference]{IEEEtran}

\usepackage[english]{babel}
\usepackage[utf8x]{inputenc}
\usepackage{amsmath}
\usepackage{graphicx}
\usepackage[colorinlistoftodos]{todonotes}
\usepackage{hyperref}
\usepackage{import}
\usepackage{cite}
\usepackage{lscape}
\usepackage{multirow}
%\usepackage{dblfloatfix}    % To enable figures at the bottom of page



%\bibliographystyle{ieeetr}
%\providecommand{\abs}[1]{\lvert#1\rvert}


\title{Parashat: Haazinu \\ \textit{Escuchen}}

% for over three affiliations, or if they all won't fit within the width
% of the page, use this alternative format:
% 
\author{\IEEEauthorblockN{Kunst, James Jules Waldemar}
\IEEEauthorblockA{Bet-Melej Haba}}
%\IEEEauthorblockA{waldemark89@hotmail.com}}


\begin{document}
\maketitle


\section{Introduction}



\section{Etapas - Nuevo Mundo}
\begin{itemize}
\item
Las ultimas 3 parasha de la Torah, son consideradas como el cierre. La conclusion.
\item
Sin embargo, hablan de la muerte de Moshe. ¿ Que significa esto?
\item 
¿Porque no sabemos donde esta enterrado Moshe?. ¿Porque la Torah lo quiso asi?
\item
Hay un Epilogo en la conclusion de la Torah?
\item
¿Cual es la diferencia de SHEMA , con HAAZINU?.¿ Ambas palabras siginifican lo mismo?
\end{itemize}





\bibliographystyle{IEEEtran}
\bibliography{reference}





\end{document}
