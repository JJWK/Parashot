
\documentclass[conference]{IEEEtran}

\usepackage[english]{babel}
\usepackage[utf8x]{inputenc}
\usepackage{amsmath}
\usepackage{graphicx}
\usepackage[colorinlistoftodos]{todonotes}
\usepackage{hyperref}
\usepackage{import}
\usepackage{cite}
\usepackage{lscape}
\usepackage{multirow}
%\usepackage{dblfloatfix}    % To enable figures at the bottom of page



%\bibliographystyle{ieeetr}
%\providecommand{\abs}[1]{\lvert#1\rvert}


\title{Parashat: Ki-Tavo \\ \textit{Cuando entres}}

% for over three affiliations, or if they all won't fit within the width
% of the page, use this alternative format:
% 
\author{\IEEEauthorblockN{Kunst, James Jules Waldemar}
\IEEEauthorblockA{Bet-Melej Haba}}
%\IEEEauthorblockA{waldemark89@hotmail.com}}


\begin{document}
\maketitle


\section{Introduction}



\section{¿Maldiciones por falta de felicidad?}

Las maldiciones, no son generales, son especificas, por ejemplo Dev 28:30:
\begin{itemize}
\item Tendras esposa, pero otro hombre estara con ella: 
\item Tendras casa, pero no viviras en ella 
\item Tendras viña pero no la disfrutaras
\end{itemize}

no son simples malos augurios, sino que involucran trabajo, mucho trabajo el cual no es aprovechado.
Pero aun queda la mas terrible de todas. Dev 28:53, comeras el fruto de tu vientre. Esta maldicion es muy detallista.

\begin{itemize}
\item
¿Que lo lleva a D's a ser tan detallista?
\item 
¿Cual es la causa de estas maldiciones?. Dev 28:47 , ¿solo por falta de felicidad?. Osea , que puedo ser que cumpla todo, pero si no soy feliz, ¿merezco este castigo? - Dev 26:11
\end{itemize}


Para responder estas preguntas, busquemos analogias con otra Parahat. En Bereshit, en el origen del hombre, pasa algo similar.
Esto quiere decir, que llegando al final de la parashat, nos conectamos con el inicio, dando a entender una repeticion o un ciclo.

Veamos la parabra, \textit{tov}, que significa \textit{bueno}. Esta palabra aparee muhcas veces tanto en la parasha, como en Bereshit. Dando a entender que todo lo que da D's en ese momento es bueno, muy bueno.
Otra palabra que aparece en ambos lugares, es \textit{Disfrutar}, D's invita a disfrutar de lo bueno que el da. Ber 2:8? 

Entonces, la humanidad tiene todo y D's  la invita a que lo disfruten. Sin embargo, la humanidad se pregunta: Acaso, ¿tengo todo?. No me falta ese arbol prohibido?. La humanidad lo toma y cae en la muerte.

En esta Parashat, suscede algo similar, estamos bajo las mismas condiciones, en la misma situacion  en la misma prueba. 
El ciclo se repite, es una nueva oportunidad. Pero, a diferencia de antes, esta vez, D's agrega una palabra mas. Que seamos agradecidos, que tengamos gozo.

Todo lo que D's nos da, son regalos para profundizar nuestra relacion, no para darlo por hecho. Sin para reconocerlo, para ser agradecidos. Y la forma de serlo, es con gozo.- que seamos felices con lo que nos da.

Existen dos tipos de deseo, el deseo bueno que nos lleva a elevarnos, el desar algo que nos haga bien, y luego esta el deseo, que destruye , el desio vacio. Este deseo, genera un descontento con lo que tenemos o recibimos, llevandonos a desar cada vez mas.


En Bereshit, la serpiente introduce el deso vacio en Eva, llevandola a creer que teniendo el 99\% de las cosas, no es suficiente, sin que solo con el 1\% que falta, sera feliz. 
Es muy dificil ser feliz, sabiendo que desamos cada vez mas. Nunca estamos llenos, justamente por este vacio. 


En cambio, el reconocimiento de lo bueno nos lleva a la felicidad. La felicidad es la reaccion al recnocimiento de que es bueno. De lo bueno que tenemos delante. 

Algo bueno  , no crea felicidad instantanea. hay un paso en el medio, que es el reconocimiento, tomar conciencia. Esta practica, nos lleva a reconocer las cosas buenas que nos rodean.

El ultimo paso, es reconocer a la persona que permitio este bien, sentirce agradecido. 

Y esto es a lo que lleva la oracion a las primicias, la cual esta al inicio de la Parasha. es esta practica lo que nos queire enselar D's.

En base   a lo anterior dicho. Si vemos las Maldiciones desde este punto, en realidad las maldiciones son consecuecia de no ser agradecidos, son maldiciones que nos ponemos nosostros mismos, al no reconocer lo bueno delante de nuestros ojos.

\begin{itemize}
\item Tendras esposa, pero al querer mas, no las disfrutaras, y se ira.
\item tendras casa , pero al querer mas, nunca estaras en ella
\item Tendras viña, pero al querer mas, nunca disfrutaras de ella.
\end{itemize}
Las maldiciones no son de D's, sino que son consecuencias.

Por que no las reconoces como algo bueno, no las disfrutas, no sos agradecido. y al final es como si no fueran tuyas.

 Y al final, el descuido de estas cosas , la consecuancia mayor, es la maldicion mas terrible de todas, que comeras a tus propios hijos. Esto es conscuencia de descuidar todo lo que posees, de decuido de ser agradecido. 
 
 En la epoca medieval, ser descuidado traia grandes consecuancias. Actualmente , ser descuidado, trae otra clase de consecuencias, que pueden ser igual de terribles.



Al final, de los arboles que estan en la Torah, cual tomarias?

Porque esta ultima palabra de, ser agradecidos,¿ no estaba en Bereshi?. Acaso,¿ D's se la olvido de decirnosla?.
No, sino que D's esperaba que la resolvieramos nosotros. Es como si les diera un problema de matematica, y lo resolvieran por si mismos, esto llevaria y demostraria que estudiaron, que meditaron, y que son maduros. Al contrario, si les doy la respuesta, no tienen ningun merito, simplemente saben el resultado son ningun ezfuerzo y se pierden muchos aprendizajes en el medio.
Esta es la principal conicidiencia con Bereshit y en esta Parashat. Ki-Tavo es elmismo problema , pero con el resultado. 
El resultado era que teniamos a aprender por nosotros mismos a ser agradecidos de todo lo bueno que D's nos da.




\bibliographystyle{IEEEtran}
\bibliography{reference}





\end{document}
