
\documentclass[conference]{IEEEtran}

\usepackage[english]{babel}
\usepackage[utf8x]{inputenc}
\usepackage{amsmath}
\usepackage{graphicx}
\usepackage[colorinlistoftodos]{todonotes}
\usepackage{hyperref}
\usepackage{import}
\usepackage{cite}
\usepackage{lscape}
\usepackage{multirow}
%\usepackage{dblfloatfix}    % To enable figures at the bottom of page



%\bibliographystyle{ieeetr}
%\providecommand{\abs}[1]{\lvert#1\rvert}


\title{Parashat: Vaishaj\\ \textit{}}

% for over three affiliations, or if they all won't fit within the width
% of the page, use this alternative format:
% 
\author{\IEEEauthorblockN{Kunst, James Jules Waldemar}
\IEEEauthorblockA{Bet-Melej Haba}}
%\IEEEauthorblockA{waldemark89@hotmail.com}}


\begin{document}
\maketitle


\section{Introduction}
\begin{itemize}
\item
Bereshit 32:4-36:43
\end{itemize}

La parashat cuenta los siguientes episodios:
\begin{itemize}
\item
Jacob lucha con un ángel
\item
Jacob se encuentra con Esav
\item
Historia de Dina
\item
Profecías
\item
Muerte de Isacc
\end{itemize}

Esta Parashat habla de dos acciones o formas de proceder de Jacob que son opuestas.
Los sabios tienen diferentes puntos de vista con respecto a Jacob.
Hay algunos sabios que intentar justificar el accionar de Jacob, es decir resaltan las cosas que hizo bien y a las que hizo mal las justifican. Otros sabios en cambio que todo lo que hizo Jacob estuvo bien, al contrario de los anteriores no aceptan que hizo algún mal, por lo que no hay que justificar nada.
Yo por mi parte voy a analizar un poco a Jacob y voy a intentar sacar alguna conclusión que este en el medio de estas dos formas de pensar.



\section{Preguntas}
Para poder analizar a Jacob me voy a hacer dos preguntas.
\begin{enumerate}
\item ¿Que significa todos los regalos que de Jacob le manda a Esav y todas las palabras que se dicen entre ellos?
\item ¿Porque Jacob se mantiene en silencio ante la violación de Dina?
\end{enumerate}



\section{Desarrollo}

\subsection{Jacob y Esav}











Imaginémonos que robamos algo muy valioso a alguien.Ya sea porque lo queríamos o lo necesitábamos, no importa.
Imaginemos en esa situación, sumado a eso, tengamos en cuenta que la persona dueña del objeto de valor sabe que fuimos nosotros y ademas sabe donde estamos. Y nosotros no podemos escapar. Que es lo primero que vamos a pensar.
Lo primero es que estoy en el horno, y lo segundo es que esa persona lo primero que quiere que hagamos es que le devolvamos el objeto. Después nos hará meter en la cárcel o nos hará pagar una fianza o nos perdonara. No importa. Lo primero que esa persona quiere es que le devolvamos lo quitado.

De esa forma se sentía Jacob, el le había robado la bendición de la primogenitura a Esav. Tenia que estar preparado para devolver lo que le había robado. 




GEn 29
hay varias explicaciones de porque llora jacob, una es que antes de conocer a rachel, estuvo pensando en su hermano Esav, y cuando se encuentra con rachel, y se da cuenta que nun ca podria tenerla, porque es pobre , y es como ver algo que nunca podra tener, y se da cuenta lo que sintio Esav en cuanto a su bendicion. 






Esav  queria que se le devuelvan la Bendicion. Jacob sabia eso, entonces cuando se entera que Esav viene a buscarlo, ¿Que es lo primero que hace? 

Le devuelve la bendicicion, ¿ como? , mandandole ganado. Y como puede ser que eso signifique que le esta devolviendo la bendicion?.
\begin{itemize}
\item Bereshit 27:27 - 28
 
Dios, pues, te dé del rocío del cielo,
Y de las grosuras de la tierra,
Y abundancia de trigo y de mosto.
\end{itemize}

Lo Jacob le envia a Esav no solo son animales, sino que si lo analizamos le manda las hembras con sus crias , es decir le esta dando basicamente una inversion. Una inversion de mucho valor economico en esos dias.

Luego de que Esav recibe los regalos  y se encuentran que es lo primero que hace Jacob?. Se inclina , y si continuamos leyendo la bendicion.

\begin{itemize}


\item Bereshit 27: 29

Sírvante pueblos,
Y naciones se inclinen a ti;
Sé señor de tus hermanos,
Y se inclinen ante ti los hijos de tu madre.
Malditos los que te maldijeren,
Y benditos los que te bendijeren.


\end{itemize}


Lo que Jacob esta haciendo, es devolviendo le la bendicion a Esav. La bendición que recibió Jacob se la esta dando a Esav. 


Y ahora que debería pasar?. Esav una vez que recibio los regalos, deberia dar su sentencia. Esav lo perdona y le besa el cuello. Y se reconcilian.


Que podemos aprender de Jacob con este episodio?. Que es una persona que hacer todo lo posible para lograr la paz. 
concluimos que aprendió de sus errores. Tuvo mucho tiempo para meditar.



La historia puede terminar aquí, felizmente, exaltando el accionar de Jacob y Esav, pero , hay otra historia,  la historia de Dina.

\pagebreak
\subsection{Historia de Dina}
La historia de Dina, si uno la lee sin prestarle atención parece que esta fuera de lugar. Arruina la paz que se gano unos versículos antes.

Repasemos, Dina es hija de Lea y hermana de Simon y Levi.

\begin{itemize}
\item La historia esta relatada en Bereshit 34\\
Resumen:\\
Dina camina y Shekel hijo del Rey Jamor la rapta y viola, luego se enamora  
y se quiere casar con ella, van en busca de Jacob y acuerdan que todo el pueblo del rey Jamor se debe circuncidar.
Al tercer dia Simon y Levi matan a todos y rescatan a Dina, finalmente Jacob los amolesta y ellos reclaman de como Jacob no va a hacer nada ante la violación de Dina.
\end{itemize}

Hay un par de  detalles muy importantes de esta historia.

\begin{itemize}
\item
Bereshit 34:1  ---Salio Dina hija de Lea...\\
Sabemos que es hija de Lea y tambien de Jacob , pero tal vez haya algo mas que nos quiera decir la Tora con este detalle.

Luego cuando el Rey Jamor se prensenta ante Jacob , tambien estan Levi y Simon. 
Jamos dice :
 
\item
Bereshit 34:5 --- jacob escucha que Dina es violada como " su hija" \\
Jacob no hace nada permanece en silencio

\item Bereshit 34:15
Simon y Levi...se refieren a Dina como nuestra hijas
Los hermanos de Dina toman el lugar del Padre.



\item 
permitiremos  que nuestra hermana sea violada?
no deberian decir... como permites que tu HIJA sea violada?
%%%%%%%%%%%%%%%%%%%%%%%%%%%%%%

\end{itemize}

%De donde aprendieron a usar el engaño simon y levi  contra el Rey Jamor?
%Hisoria de engaño de laban y Jacob, Laban engaña-
%los hijos de Jacob aprendieron el engaño de Laban.?
%pero Jacob tambien se lo hizo a esav.
%es una cadena de engaños. entre generaciones. 
%pero que los llevo a los hijos de Jacob a accionar de esta forma? 

Que motivo a los hijos de Jacob a accionar de forma engañosa contra el Rey Jamor?
Que los motivo?

Una forma de ver todo este episodio es Jacob no reacciona porque Dina es hija de su esposa no favorita, es decir o es hija de Rachel. Es por eso que la Tora nos enfatiza que Dina la hija de Lea.
Los hijos se dan cuenta del favoritismo y reaccionan a eso.
Se dan cuenta del favoritismo.
En conclusión, estamos viendo un mal accionar de Jacob. 
Sus hijos sienten descartados, se sienten como hijos de segunda, por ser hijos de Lea.
y hacen todo lo necesario por defender a su hermana, toman el lugar de su padre. 

Cuando el Rey Jamor se les presenta Simon y Levi los engañan.

Donde vemos una hisoria parecida?

Veamos la juventud de Jacob, cuando vivía con su padre Isacc.
Entre Jacob y Esav, ¿Quien fue el hijo Favorito?, Esav .
Jacob tambien sufrio el daño causado por el favoritismo de Isacc.
y a causa de eso, engaña a su Padre. 

Tal cual sucedio con Simon y Levi.

Un mal que encadena otro mal.

Y un mal que se va a volver a repetir con la historia de Iosef.

Analizando todo lo anterior, la historia de Dina se conecta con la historia de Iosef.
porque anticipa el favoritismo de Jacob , anticipa el error de Jacob, y como reaccionan sus hijos ante ese error.
que engañan y sacan lo peor de si.


Que tanto daño puede causar ese accioar en la familia de Jacob y seran capaces los hermanos de Iosef a enfrentar a Jacob.


Jacob se reconcilia con su hermano Esav , pero podrán los Hijos de Jacob reconciliarse entre ellos?




\section{Conclusion}
\begin{itemize}
\item
Jacob era una persona de este mundo, lo que la hace estar atada a ciertos pensamientos 
culturales de la época con los que actualmente podríamos estar en desacuerdo.
La sociedad era muy machista, tenia una actitud de favoritismo entre los hijos.
\item
Jacob era una persona entrada en años y es dificil hacerle cambiar la forma de pensar.
\item
Una persona con defectos.
\item
Una persona imperfecta.
Con defectos como cualquiera de nosotros. 

\end{itemize}

Sin embargo y a pesar de todo
\begin{itemize}
\item
Jacob era una persona bendecida por D's.
\item
Era descendiente de Abraham 
\item
Un hijo de D's
\item
Y fue capaz de reconciliarse con su hermano
\end{itemize}


Jacob por causa de sus defectos, sufre, como sufrio , con Esav hasta que se reconcilio y sufrio aun mas con Iosef
hasta que los hermanos se reconciliaron.
Lo importante es que Jacob era una persona como nosotros , capaz de hacer tanto buenas como malas deciciones , pero es capaz de meditar y buscar todos los medios para compensar ese error. 
En el Judaimos Jacob es considerado como el patriarca de la verdad.  ¿Pero como es posible conociendo la cantidad de errores que tenia? Jacob fue aprendiendo de sus erroes hasta convertirse en una persona de verdad

\begin{itemize}
\item Jacob no  nace como una persona de verdad.
\item Jacob se desarrolla hasta ser una persona de verdad.
\item Nadie nace fuerte sino se hace fuerte.
\item Nadie nace sabio, sino que se hace sabio.
\end{itemize}


\bibliographystyle{IEEEtran}
\bibliography{reference}





\end{document}
