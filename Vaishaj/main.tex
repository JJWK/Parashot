
\documentclass[conference]{IEEEtran}

\usepackage[english]{babel}
\usepackage[utf8x]{inputenc}
\usepackage{amsmath}
\usepackage{graphicx}
\usepackage[colorinlistoftodos]{todonotes}
\usepackage{hyperref}
\usepackage{import}
\usepackage{cite}
\usepackage{lscape}
\usepackage{multirow}
%\usepackage{dblfloatfix}    % To enable figures at the bottom of page



%\bibliographystyle{ieeetr}
%\providecommand{\abs}[1]{\lvert#1\rvert}


\title{Parashat: Vaishaj\\ \textit{}}

% for over three affiliations, or if they all won't fit within the width
% of the page, use this alternative format:
% 
\author{\IEEEauthorblockN{Kunst, James Jules Waldemar}
\IEEEauthorblockA{Bet-Melej Haba}}
%\IEEEauthorblockA{waldemark89@hotmail.com}}


\begin{document}
\maketitle


\section{Introduction}
\begin{itemize}
\item
Bereshit 32:4-36:43
\end{itemize}


Hay muchas formas de analizar a los pratriarcas,
hay algunos sabios que intentar justificar el accionar de Jacob, es decir que hizo algunas cosas mal, pero estaban justificadas.
Otras que reclaman que todo lo que hizo Jacob estuvo bien, y lo exaltan como alguin perfecto.




La parasha tiene como eje central a Jacob, y su forma de reaccionar en diferentes ocaciones.
Vamos a estudiar las acciones de Jacob, y voy a sacar una conclucion propia. 





\section{Preguntas}

\begin{enumerate}
\item Que significa toda la reaccion de Jacob ante Esav?
\item Como reacciona Jacob ante lo que le pasa a Dina y porque?
\end{enumerate}



\section{Desarrollo}

\item 
Que significa la reaccion de Jacob ante Esav?

Si consideramos que Jacob le quito algo a Esav, entonces los mas obvio es pensar que Esav quiere de vuelta lo que 
Jacob le quito, esto es la primogenitura.



GEn 29
hay varias explicaciones de porque llora jacob, una es que antes de conocer a rachel, estuvo pensando en su hermano Esav, y cuando se encuentra con rachel, y se da cuenta que nun ca podria tenerla, porque es pobre , y es como ver algo que nunca podra tener, y se da cuenta lo que sintio Esav en cuanto a su bendicion. 






\item
Historia de Dina







\begin{enumerate}
 \item
\end{enumerate}




\section{Conclusion}

Jacob era una persona de este mundo, lo que la hace estar atada a ciertos pensamientos 
culturales de la epoca con los que actualmente podriamos estar en desacuerdo.
La sociedad era muy machista, tenia una actitud de favoritismo entre los hijos.


UNa persona entrada en años tal vez ear dificil hacerle cambiar la forma de pensar.

Una persona con defectos.



Una persona imperfecta

con defectos como cualquiera de nosotros. 

sin embargo

sin embargo Jacob era una persona bendecida por D's.

Era decendiente de Abraham 


Un hijo de D's

Jacob por causa de sus defectos, sufre, como sufrio , con Esav hasta que se reconcilio y sufrio aun mas con Iosef
hasta que los hermanos se reconciliaron.





Lo importante es que Jacob era una persona como nosotros , capaz de hacer tanto buenas como mals deciciones , pero es capaz de meditar y buscar todos los medios para compensar ese error. 

Jacob no nace como una persona de verdad.
Jacob se desarrolla hasta ser una persona de verdad.
Nadie nace fuerte sino se hace fuerte.
Nadie nace sabio, sino que se hace sabio.



\bibliographystyle{IEEEtran}
\bibliography{reference}





\end{document}
